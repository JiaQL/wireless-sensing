\section{Significant challenges and potential solutions}
\textbf{Multi-path effect:}  Due to the inherent propagation nature of wireless signals~(they can be reflected, refracted and diffracted off surrounding objects), multiple copies of the same signal may reach the receiver with different phase delays and power attenuation. This phenomenon is called the multi-path effect. Especially in indoor environments, the ubiquitous multi-path effect drastically affects the behavior of the signals, thus makes it challenging to obtain the fundamental information (RSSI, phase, AoA, ToF, etc) for accurate target sensing. While recently literature has proposed to exploit multiple links for nullify the static noises, or select optimal sub-carriers that are resistant to multi-path effect for sensing task, these schemes have poor adaptability to environmental changes. Therefore, there is a need for robust and effective multi-path and noise suppressing algorithms in accurate wireless sensing. We suggest that beam forming and SAGE algorithm would be the potential candidates to deal with this challenge.

\textbf{Time asynchronous:} To the best of our knowledge, directly using the timestamps reported by commodity devices such as Wi-Fi cards can obtain TOF at a granularity of several nanoseconds, leading to tracking error of a few meters. Although some super-resolution algorithms are applied to obtain finer TOF estimates, the underlying assumption is that all APs (including transmitters and receivers) are time synchronized. This assumption however is unrealistic for commodity Wi-Fi or other devices. Thus dealing with the time asynchronous problem is a crucial as well as challenging issue in wireless sensing. Connecting the transceiver pair by the synchronization line is a effective solution, but it is limited to scenarios where transceivers are closer to each other. Hence, a wireless synchronization scheme is urgently needed.

\textbf{Deployment issue:} To the best of our knowledge, the non-contact sensing performance is highly relevant to the deployment when the transmitter and the receiver are placed separately, due to the presence of Fresnel zones. The optimal transceiver placement would contribute to high sensing accuracy, while an unsuitable placement would lead to large amount of blind spots for sensing. So the deployment issue should be taken into account in real applications. Of course, we need to pay attention to an implicit trade-off between improving the sensing accuracy and resolving the inconveniences. Since the best deployment plan is always at the expense of more restrictions, so that brings much inconvenient experience for user. Therefore, it is recommended to choose the optimal deployment or random deployment according to the specific application, and make a good compromise between practicality and sensing performance.

\textbf{Transfer problem:} Many learning based sensing methods can achieve high accuracy without requiring intensive communication links, and can work with even a single transceiver pair, which ensures low hardware cost and great convenience. However, one major unsolved issue is that the learned patterns can hardly adapt to the changes of environment and time. For long-term or varying scenarios, the sensing accuracy would decline seriously if we still utilize the original patterns, and constant updating patterns comes with a huge training cost. Recent works attempt to solve this problem by deep learning technology that can extract hidden useful features of higher dimensions, while these methods lack theoretical basis. To make sensing more convincing, it is better to combine learning with theoretical models by excavating and analyzing the physical nature of signal changes.

\textbf{Tiny signal extraction:} Some interfaces realize their sensing objective by leveraging the reflected signal caused by the target. Nevertheless, the wireless signal arriving at receiver is always a superposition of  direct path and various reflected paths. In many cases, the target reflected signal is quite subtle compared to the signal from direct path and strong reflectors, so it is very challenging to separate the weak component from the mixed signal. Traditional solutions generally suppress undesired paths by prior measurement, while they are still unable to thoroughly attain the desired reflected component. The latest work introduce a new idea that conducts nonlinear frequency shifting to the signal reflected off target, this makes it different from the other interfering signals, thus it can be separated. However, this approach still limited to contact sensing. To achieve tiny signal extraction in more complex non-contact scenarios, we believe that the multiple paths approximate estimation and blind-source separation algorithms can serve as the promising solutions.

\textbf{Evaluation standardization:} Currently, there is no standard set of specifications that can serve as a guide for designing sensing techniques. There is no single wireless technology that is widely accepted as the main technology for future sensing systems. As evident from our discussion on the proposed systems in the previous sections, a number of different technologies and techniques have been used for the purpose. However, most of the systems are disjoint and there is no ubiquitous system that currently exists. This poses significant challenges. Therefore, we believe that there is a need for proper standardization of sensing. Through standardization, we can set the specifications and also narrow down the technologies and techniques that can satisfy the regular evaluation metrics. Furthermore, there is a need for creating a universal benchmarking mechanism for evaluating an sensing system.

\textbf{Public data sets: }In order to facilitate communities to restore the published system and ensure the feasibility of the system, opening data set is becoming a trend in the academic community. However, from the authors' point of view, it may be inconvenient to publish all data sets in a time,due to the consideration of later research extensions. We suggest that a compromise way is to expose source code and some data sets.

