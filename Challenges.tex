\section{Challenges of Wireless Sensing}
This article has by and large been very upbeat about wireless sensing. However, there are a number of hurdles to overcome to make it a
practical reality.

\subsection{Multipath Effect}

As wireless signals can be reflected, refracted and bounced off surrounding objects like walls and furniture in an indoor environment,
multiple copies of the same signal may reach the receiver with different phase delays and power attenuation. This phenomenon is known as
the multipath effect. Multipath makes it difficult to obtain clear signal information (\RSSI, phase, \AoA, \TOF, etc.) which is required
for achieving accurate sensing. While recently literature has proposed to exploit multiple links for cancel the static noises, or select
optimal sub-carriers that are resistant to multipath effect for sensing task, how to make sure the scheme can be portable to different
environments remains an open problem. There are other schemes for dealing with the multipath effect, but they often require a large
bandwidth or antenna array. This assumption reduce the practicability of the approach.


Many wireless sensing methods work by observing, from the receiver side, how the signals sent by the transmitter are affected by the target
object. Often, the reflected signals are weaker compared to those bounced from non-target objects (e.g., walls or furniture), making it
difficult to track the desired signals. Finding a way to cancel the impact of other surrounding objects would be helpful, but existing
solutions generally assume the environment is static (i.e., no other people movement around or no door opening and closing). Therefore,
there is a need for effective multipath and noise suppressing algorithms.


\subsection{Clock Synchronization} Wireless sensing systems often use multiple devices, such as transmitters and receivers, to track
objects and activities. These devices often need to exchange information at a high frequency. Precise clock synchronization is essential
for good performance because the time is used to calculate wireless channel information such as the phase, and an error of nanoseconds can
translate to a localization error of a few meters. However, obtaining synchronized time at a fine-grained level among computing multiple
devices is challenging due to the differences in the hardware clocks. One solution is to using a wire to synchronize the clocks between two
devices, but the effectiveness of the solution decreases as the distance between the devices increases. This limits the applicability of
the approach.

\subsection{Deployment Cost} The performance of a wireless sensing system highly depends on how the transmitters and receivers are placed
and the environment the sensing model is tuned for. Determining the optimal device deployment is challenging due to the subtle interactions
of the signal coverage and the multipath effects~\cite{wang2016human}. The optimal deployment also changes if the environment has changed,
e.g., adding or removing some furniture. Such a change often requires recollecting the training samples to update the sensing model, which
is often associated with intensive human involvement at a high cost. A recent work addresses the issue by using machine learning to
generate synthetic training examples~\cite{CrossSense}. While this work was able to reduce the cost for collecting training data, it does
not eliminate it. Therefore, the issue of deployment cost remains.



%\paragraph*{Generalization problem} Many learning based sensing methods can work effectively with even a single transceiver pair.
%This reduces the hardware cost. However, one major unsolved issue is that the learned model can hardly adapt to the changes of environment
%and time. For long-term or varying scenarios, the sensing accuracy would decline seriously if we still utilize the original patterns, and
%constant updating patterns comes with a huge training cost. Recent works attempt to solve this problem by deep learning technology that can
%extract hidden useful features of higher dimensions, while these methods lack theoretical basis. To make sensing more convincing, it is
%better to combine learning with theoretical models by excavating and analyzing the physical nature of signal changes.

%\subsection{Tiny signal extraction} Some interfaces realize their sensing objective by leveraging the reflected signal caused by the
%target. Nevertheless, the wireless signal arriving at receiver is always a superposition of  direct path and various reflected paths. In
%many cases, the target reflected signal is quite subtle compared to the signal from direct path and strong reflectors, so it is very
%challenging to separate the weak component from the mixed signal. Traditional solutions generally suppress undesired paths by prior
%measurement, while they are still unable to thoroughly attain the desired reflected component. The latest work introduce a new idea that
%conducts nonlinear frequency shifting to the signal reflected off target, this makes it different from the other interfering signals, thus
%it can be separated. However, this approach still limited to contact sensing. To achieve tiny signal extraction in more complex non-contact
%scenarios, we believe that the multiple paths approximate estimation and blind-source separation algorithms can serve as the promising
%solutions.

\subsection{Evaluation Standard}
Currently, a proposed sensing technique is typically evaluated on bespoke datasets and the experiment is conducted in specific environments
and settings. Like any engineering subject, the lack of evaluation standards and common datasets make it difficult to evaluate the
effectiveness and generalization ability of different approaches. Many existing approaches in the field are built upon machine learning
techniques. Peer review of a machine learning approach is difficult. The black-boxing mechanism prevents the quality of the model from
being questioned unlike analytical methods. As a result, reviewers now have to scrutinize that the experiments were fairly done. This means
all the implementation and test data must be publicly available for scrutiny.

We see the research community is pushing reproducibility by e.g., encouraging authors to opensource their code and datasets and to
participant in artefact evaluation. We hope such movements can result in some widely acceptable evaluation standards and public datasets.


 %Currently, there is no standard set of specifications that can serve as a guide for designing
%sensing techniques. There is no single wireless technology that is widely accepted as the main technology for future sensing systems. As
%evident from our discussion on the proposed systems in the previous sections, a number of different technologies and techniques have been
%used for the purpose. However, most of the systems are disjoint and there is no ubiquitous system that currently exists. This poses
%significant challenges. Therefore, we believe that there is a need for proper standardization of sensing. Through standardization, we can
%set the specifications and also narrow down the technologies and techniques that can satisfy the regular evaluation metrics. Furthermore,
%there is a need for creating a universal benchmarking mechanism for evaluating an sensing system.
%
%In order to facilitate communities to restore the published system and ensure the feasibility of the system, opening data set is becoming a
%trend in the academic community. However, from the authors' point of view, it may be inconvenient to publish all data sets in a time,due to
%the consideration of later research extensions. We suggest that a compromise way is to expose source code and some data sets.
