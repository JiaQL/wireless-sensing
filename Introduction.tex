\section{Introduction}
The development of wireless devices has resulted in numerous wireless sensing technologies which enable a variety of applications such as indoor navigation, smart homes, fitness tracking and security check. In recent years, especially non-contact wireless sensing has attracted much attention from both the research community and the industry, due to its convenience of without instrumenting user's body. Many researches have answered how to identify object's location, shape, material, behavior, etc. The basic idea of these designs is to utilize signal processing to construct a mapping pattern or theoretical relationship between the target information and the signal characteristics.

To turn these designs into practical systems, series challenges need to be tackled. Typically, (i) the defects of cheap devices can lead to nonnegligible error of signal information~(phase), (ii) environmental changes can alter the regular signal patterns, and (iii) multi-path effect would greatly reduce the estimation resolution of some fundamental parameters for sensing. These always make the system performance extremely poor. To address these issues, many efforts has been made, including: (i) calibrating the inaccurate phase information through comparing with it with a real value obtained in advance, (ii) canceling background noise using signals from multiple links, and (iii) improving the resolution of parameter estimation by splicing bandwidth or constructing virtual array. Nowadays, with the popularity of artificial intelligence, deep learning technology is gradually integrated into signal processing to extract more complex and effective features for sensing task. This drives great success in wireless sensing applications.

In the rest of this paper, we first review the state-of-arts of wireless sensing. Then we conclude the key challenges of this field and suggest promising solutions for them. Finally, we propose guiding insights into future trends.
