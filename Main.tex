%
\documentclass[conference]{IEEEtran}
\ifCLASSINFOpdf

\else \fi

\hyphenation{op-tical net-works semi-conduc-tor}

\usepackage[square, comma, sort&compress, numbers]{natbib}
\usepackage{graphicx}
\usepackage{subfigure}
\usepackage{caption}
\usepackage{mathrsfs}
\usepackage{algorithm}
\usepackage{algpseudocode}
\usepackage{amsthm}
\usepackage{amsmath}
\usepackage{graphics}
\usepackage{epsfig}
\usepackage{amssymb}
\usepackage{multirow}
\usepackage{setspace}
\usepackage[table]{xcolor}
\usepackage{breqn}
\usepackage{url}
\usepackage{caption}
\usepackage{color,colortbl}
\usepackage{xspace}
\usepackage{booktabs}


\newtheorem{assump}{Assumption}
\newtheorem{axiom}{Axiom}
\newtheorem{claim}{Claim}
\newtheorem{conj}{Conjecture}[section]
\newtheorem{crit}{Criterion}
\newtheorem{theorem}{Theorem}
\newtheorem{coro}{Corollary}[theorem]
\newtheorem{definition}{Definition}
\newtheorem{example}{Example}
\newtheorem{fact}{Fact}
\newtheorem{hypo}{Hypothesis}
\newtheorem{lemma}{Lemma}
\newtheorem{obsv}{Observation}[theorem]
\newtheorem{prin}{Principle}
\newtheorem{prob}{Problem}
\newtheorem{ppty}{Property}
\newtheorem{propo}{Proposition}
\newtheorem{protocol}{Protocol}
\newtheorem{remark}{Remark}
\newtheorem{test}{Test}
\newtheorem{transform}{Conversion Algorithm}

\newcommand{\tabincell}[2]{\begin{tabular}{@{}#1@{}}#2\end{tabular}}
\renewcommand\d{\mathop{}\!\mathrm{d}}
\newcommand{\systemname}{WiMi\xspace}

\newcommand{\CSI}{\texttt{CSI}\xspace}
\newcommand{\RSSI}{\texttt{RSSI}\xspace}
\newcommand{\RF}{\texttt{RF}\xspace}
\definecolor{Gray}{gray}{0.9}

\newcommand\FIXME[1]{\textcolor{orange}{FIX:}\textcolor{orange}{#1}}
\newcommand\red[1]{\textcolor{red}{\textcolor{red}{#1}}}


\begin{document}

%
\title{Sensing Our World Using Wireless Signals}


\author{}

\maketitle

\begin{abstract}
In the last few years, pervasive wireless signals have been established as a powerful medium for ubiquitous sensing. %By carefully processing the variation of the signal reflected from the human target, rich information ranging from coarse-grained body activity to fine-grained hand gesture and even vital signs can now be accurately  sensed without any device attached to the human body. 
Wireless sensing quickly becomes an active cross-disciplinary research area which involves wireless communication, signal processing, human-computer interaction, machine learning and even hardware design. This article serves as an introduction to this exciting area and presents the key technologies enabling it. This article discussed the state-of-the-arts achievements in this field and also the unaddressed research challenges.
\end{abstract}


\IEEEpeerreviewmaketitle

\section{Introduction}
%Wireless signals, such as WiFi and the radio frequency (\RF), have emerged as a cheap yet powerful medium for information sensing. By
%measuring how the wireless signal is affected by surrounding objects and human activities, tasks like gait identification~\cite{}, gesture
%recognition~\cite{}, activity recognition~\cite{}, and vital sign monitoring~\cite{} can be possible.
%
%
%The desire for employing wireless signals for sensing could date back to the 19th century  for the discovery of X-rays for object
%recognition~\cite{} and the development of military radar and sonar systems for tracking large metallic objects in open
%spaces~\cite{Charles Samuel Franklin 's development of first practcial radar}. These sensing systems, however, have to rely on expensive specialize hardware and be operated by professionals, which thus put they out of the reach of ordinary people.
%
%
%Recently, it has been shown that wireless sensing built upon consumer-grade devices such as smartphones and wireless routers can be used to
%precisely track human activities in an indoor environment. Such technology has the advantages of being low-cost as it requires as few as
%two wireless routers to operate, not requiring instrumenting the user, being less privacy intrusive compared to other infrastructure-based
%solutions like video-based solutions~\cite{}, and is safer to use on a regular basis compared to other alternatives like X-rays~\cite{}. It
%offers an inexpensive way for bringing activity sensing into everyday life -- which not only makes ever sensing a tantalizing reality but
%could also open up new possibilities for innovative applications.
%
%
%As we will show in this article, while there are still many challenges ahead, consumer-grade wireless sensing has moved from a research
%niche to a mainstream activity. This is, in fact, a dynamic field looking at subjects as diverse as smart-home personalization~\cite{} and
%fall monitoring~\cite{} to emotion detection~\cite{}. In this article we aim to demystify this promising technique, outline the challenges
%it is facing, and show that this is a trustworthy and exciting research direction.
%
%
%
%
%The remainder of this article is structured as follows. We first give an intuitive view for the wireless sensing working mechanism  in
%Section~\ref{sec:mechanism}. \FIXME{We then describe xx, bla bla...}. We discuss the challenges and limitations for wireless sensings, as
%well as open research directions in Section xx before we summarise and conclude in Section \FIXME{xx}.



Wireless technologies have achieved a great success in data communication, changing our lives in every aspect in the last two decades. 
%Sensing technologies play a key role in the internet of things~(IoT), changing our lives in every aspect. 
%Recently, Artificial Intelligence enters our everyday lives at an ever increasing pace
Recent years have also witnessed a surge in Artificial Intelligence, allowing the software to learn automatically from patterns or features in the data by analyzing massive amounts of data with intelligent algorithms. %Great success have been made both in respects of accuracy and robustness. 
The large amount of data comes from all sources, including sensors deployed around us such as surveillance cameras and wearables worn by us. However, camera-based solutions bring in severe privacy concerns and the elderly are usually reluctant to wear wearables.   

%In particular, wearable sensing systems hold robust performance, however they are inconvenient sometimes since the devices that users have to wear~\cite{Pinpoint}. In contrast, vision-based sensing using cameras has revolutionized the field of human-computer interaction via enabling 3D tracking without instrumenting the object~\cite{Cruz2015Quantification}. Yet these systems have the fundamental limitations of requiring obstacle-free and breaching human privacy potentially.

Recently, pervasive wireless signals such as \WiFi, \RFID and sound, have emerged as a powerful medium to sense the human target and also our surrounding environment in a contactless manner. 
The history of using wireless signals for sensing dates back to the 19th century when people utilized X-rays for imaging~\cite{Suzuki1996} and radar and sonar systems for tracking objects~\cite{Au1988Sonar}. Though powerful, these sensing systems employ expensive specialized hardware and need to be operated by professionals, putting them out of the reach of ordinary people.

The latest research exploits wireless signals generated by cheap commercial off-the-shelf (COTS) hardware such as smartphone, \WiFi access point and \RFID reader for sensing. %demonstrated to be able to sense the human activity, image the target, monitor the subtle respiration movement and identify the material type of objects as shown in Fig.~\ref{fig:scenarios}.  
With the subtle signal variations caused by the target, the target's rich information can be sensed without any device attached to the target.  Contactless wireless sensing has enabled a large variety of exciting new applications including indoor localization~\cite{Arraytrack,
Tagoram}, activity/gesture recognition~\cite{Wang2015Understanding, wang2016human}, fall detection~\cite{wang2017wifall}, respiration monitoring~\cite{Smart-homes}, emotion sensing~\cite{Zhao2017Emotion},  material identification~\cite{Tagscan, LiquID}, room layout mapping~\cite{Lu2018}, imaging~\cite{zhao2018rf}, etc. Some of the applications enabled by ubiquitous commodity WiFi hardware are shown in  Fig.~\ref{fig:scenarios}. %  are shown in Commodity WiFi hardware has been demonstrated to be able to sense the human activity, image the target, monitor the subtle respiration movement and identify the material type of objects as shown in Fig.~\ref{fig:scenarios}.  
We envision wireless sensing has a tremendous potential to be widely adopted in many of our everyday applications in the Era of IoT.



% track objects and human activities in an indoor environment, as shown in Fig.~\ref{fig:scenarios}. Such technology has the advantages of being low-cost as it
%requires as few as two wireless routers to operate, not requiring instrumenting the user,
%Contactless wireless sensing exhibits clear advantage of less intrusive compared to wearable-based sensing and does not compromise people privacy compared to video-based solutions~\cite{Cruz2015Quantification}. %Compared to the expensive and is safer to use on a regular basis compared to other alternatives like X-rays~\cite{De2013B}.
 % capabilities into everyday life, making
%ever sensing a tantalizing reality.

As we will show in this article, while there still remain unaddressed challenges, wireless sensing has achieved promising progresses and we believe this is an exciting and at the same time rewarding direction for researchers to explore.

%with CTOS hardware has moved from a research niche to a mainstream activity. This is, in fact, a dynamic field looking at subjects as diverse as smart-home personalization~\cite{vasisht2016decimeter} and
%fall monitoring~\cite{wang2017wifall} to emotion detection~\cite{Zhao2017Emotion} and vital sign monitoring~\cite{Smart-homes}. In this article we aim to demystify this promising
%technique, outline the challenges it is facing, and show that this is a trustworthy and exciting research direction.


\begin{figure} [t!]
\centering
\includegraphics[width=0.5\textwidth]{figures/scenarios.pdf}
\caption{Examples of wireless sensing.}
\label{fig:scenarios}
\end{figure}



\section{Working Mechanism of Wireless Sensing\label{sec:mechanism}}
\begin{figure} [t!]
       \centering
        \subfigure[]{
                \includegraphics[width=0.22\textwidth]{figures/setup_example.pdf}
                \label{fig:scenario_1}
        }
        \subfigure[]{
                \includegraphics[width=0.22\textwidth]{figures//diff.pdf}
                \label{fig:csi_diff}
        }%
     \caption{The working mechanism of wirless sensing. (a) shows a typical set up where two wireless devices are deployed
     to collect the \CSI values when a user walks pass the devices. (b)  shows the measured \CSI  for two
     individuals, where the resulted \CSI amplitudes are relatively consistent for the same person across multiple walks, but are sufficiently different for different people.}
     \label{fig:csi_demo}
\end{figure}


To illustrate how wireless sensing works, consider the gait identification scenario depicted in Figure~\ref{fig:scenario_1}. The sensing
task in this example is to identify which of a set of known users has walked pass the scene. Knowing this information allows one to -- for
example -- personalize the light setting of a smart home. Here, two wireless routers (a sender and a receiver) are used to measure how the
user's movement affect the wireless channel metric, such as the channel state information (\CSI) or received signal strength indicator
(\RSSI). The idea is that the wireless signal will bounce off the wall, furniture, and human body, and the unique pattern of a person’s
activity (walking in this example) would lead to a unique pattern in the wireless signal; by measuring how the wireless signal is affected
by the human activity and comparing the measurement against some pre-collected fingerprints or training data, one can infer what activity
has been performed and by whom.

Figure~\ref{fig:csi_diff} shows the measured \CSI amplitudes of two users in our scenario. In this case, each user walked through our scene
five times and the \CSI amplitude per walk is shown. This figure suggests that WiFi channel metrics can be a useful means for user
identification, because the channel metric measurements for the same person is relatively consistent but is sufficiently different for the
two  people.

\subsection {Choices of Wireless Sensing Signals}


As summarized in Table~\ref{Tab1}, a large variety of wireless signals have been employed in prior work for information sensing. These include
\WiFi, \RFID, acoustic, \LoRa, visible lights, 60GHz and ultra-wideband (\UWB). Signals like
WiFi, acoustic, and visible lights are readily available at COTS devices such as smartphones and light bulbs,
while others like \UWB and 60GHz often require dedicated hardware to operate.



Which signal is best for a sensing task, is the \$64,000 question. The answer is, it depends. For example, while \WiFi is ubiquitous indoors, it is limited in the communication range~(15-50m) and consumes a significant amount of power. In contrast, \LoRa
supports a much longer range up to 15km and needs much less power to operate, but it has a much smaller bandwidth and is more
sensitive to multipath and interference. %Different technologies also operate at different carrier frequencies and offer a variety of sensing metrics.
On the other hand, acoustic signals can achieve a surprisingly high sensing accuracy at millimetre level~\cite{LLAP} due to the much smaller propagation speed in the air. However, the sensing range of acoustic signals is usually below 1m.
%Some of the signals like \FMCW radar can provide an accurate tracking capability but is more expensive to deploy.
It is thus important to note
that there is no ``one-size-fits-all" sensing signal, and the choice depends on the trade-off between the precision requirements, the
deployment environment, and the budget.

\section{Wireless Sensing Applications}
Wireless sensing has been demonstrated to be effective on a wide range of tasks from localization onto activity recognition and material
identification.


\subsection{Localization} Target localization and tracking are two popular sensing applications. \WiFi and \RFID are the predominate schemes used for this task, due to the low-cost and wide-scale deployment of the schemes.
Acoustic signals have also been employed for localization and tracking, thanks to its slow propagation velocity and robustness to
environment changes.

 Localization can be achieved by using wireless fingerprints or a predictive model. A fingerprint-based approach works
by first collecting the wireless fingerprints of a set of carefully chosen locations. During deployment, it compares the measured wireless
channel metric (e.g., \RSSI or \CSI) against the pre-collected fingerprints, and then uses the most-similar fingerprint to infer the
location of the target object. Such an approach has a low deployment cost, but its accuracy is susceptible to the environmental change and
indoor multipath.

As an alternative, predictive models learn the correlation between the wireless signal characteristics and the indoor locations. Such
approaches can be categorized into two groups. The first is to use the arrival-of-angle (\AoA) or time-of-flight (\ToF) of the wireless
signals for location estimation. The second is to exploit the signal energy attenuation for location estimation. Predictive models have the
advantage of having a better generalization ability over the fingerprint counterparts, i.e., it can estimate arbitrary locations of the
target environment.

However, predictive models are not a panacea and often suffer from hardware imperfections (e.g., it is difficult to obtain an accurate
\ToF) and multipath effects (which make it hard for obtaining good signal attenuation). They also have the difficulties for determining the
target's initial position. Several countermeasures have been proposed to address these issues. These include (i) using pre-collected
fingerprints~\cite{Wang2016D}, directly modeling the phase differences~\cite{LLAP},  information obtained at multiple signal
frequencies~\cite{RFind, CAT, Strata}, or multiple tracking devices~\cite{BeepBeep} to improve the information gain; (ii) using multiple
antennas~\cite{Arraytrack, Spotfi} to cancel the multipath effect or directly modeling the multipath profiles~\cite{PinIt}; and (iii)
leveraging the virtual antenna array to obtain the initial position~\cite{Tagoram}.





\subsection{Activity and Behavior Recognition}
Human activity and behavior recognition is another area where wireless sensing has demonstrated great potential.

Some of the early work of the area use multiple transmitters and receivers to construct a 3D lattice of wireless links to identify if there
is a human movement and if so what is the user's location. Efforts have been made to use a single device equipped with multiple antennas to
build a practical solution. For examples, WiSee~\cite{WiSee} was able to send signals from outside a room to track human movements; and WiTrack
shows that is possible to recognize hand gestures from signal perturbations. A more recent work illustrates the possibility for using
wireless signals for measuring human's breathing and heart rate~\cite{Smart-homes}. The idea is that the chest movement caused by
breathing would alert the signal reflections, which in turn allows one to capture the breathing event and count the heart rate with enough
accuracy. By precisely counting the heart beats, one can even predict a user's emotion~\cite{Zhao2017Emotion}.

Machine learning based classifiers have been widely used in activity recognition. The effectiveness of such schemes in general depends on
the quality of the training data and features for capturing the characteristics of the activity. As a result, some of the recent research
efforts turn into seeking ways to automatically identify useful features~\cite{CrossSense} or reduce the human involvements (and thus the
 cost) for generating high-quality training data~\cite{zhao2018rf}.





\subsection{Target Imaging and 3D Reconstruction}

Target imaging is another area that has attracted extensive attentions. Traditional imaging systems like Radar, X-rays, CT employ dedicated
hardware with high frequency, large bandwidth and antenna arrays, which are extremely expensive and usually large in size. Hence, some
researchers attempt to use \RF signals to image a target' shape. Tagscan~\cite{Tagscan} leverages each propagation distance inside a target
represents one piece of target width information at one angle, by stitching together the width information at many angles can obtain the
horizontal cut image of the target. However, this approach needs a robot with antenna moves around the target, which extremely limits the
system's practicability. To deal with this issue, RF-Capture~\cite{Adib2017Capturing} utilizes \FMCW signal to reconstruct the skeleton of
a human target behind a wall by extracting the signal reflections from human body parts. However, this system exhibits some limitations.
For example, it needs target starts by walking towards the device, then it only adopts a simple model for the human body for segmentation
and skeletal stitching, which only captures a very coarse-grained target image. To further deal with these problems,
RF-Pose3D~\cite{zhao2018rf} infers 3D human skeletons by incorporating RF~(\FMCW) signal and visual stream. It first leverages deep CNN to infer
the person's 3D skeleton, and then uses the visual stream to label different target skeleton. Finally, this system can track each key point
on the human body with a cm-level accuracy.


\subsection{Object Material Identification}
Material identification is another recent breakthrough for consumer-grade wireless sensing. This task is important for many daily
applications such as detecting concealed weapons and determining food deterioration.

Existing material identification systems such as Radar, X-rays, CT, MRI and B-scan ultrasonography use special hardware with high
frequency, large bandwidth and antenna arrays. They are expensive to purchase and operate. Some of the recent work utilize the low-cost
wireless signals like \RFID to sense the target materials and achieve impressive results. TagScan~\cite{Tagscan} detects the change of the
\RSSI and phase when the \RF signal penetrates through the target. When evaluating on 10 liquids, TagScan was able to successfully identify
over 94\% of the test objects and tells the different between Coke and Pepsi. Such an approach offers a low-cost alternative, but due to
the limitation of the technology, it is not expected to be used at security-critical scenarios like airport security check.





\subsection{Vibration Detection}   Vibration detection plays a vital role in many applications, such as malfunction detection in electronic instrument,
monitoring the displacement in vehicle, etc. Traditional vibration sensing schemes require dedicated sensors, which are very expensive. And
most of them have a limit monitoring area.

To avoid these drawbacks, TagBeat~\cite{Tagbeat} proposes to utilize commodity \RFID to achieve sub-millisecond vibration sensing. In order
to magnify tiny vibration signal induced by micro-vibration this paper introduced a simple mathematical methodology to multiplies the
vibration amplitude by a constant. Also, they proposed a compressive sensing model to reconstruct the vibration signal to overcome the
challenge of detecting high-frequency vibration with a limited sampling rate.


\section{Challenges of Wireless Sensing}
This article has by and large been very upbeat about wireless sensing. However, there are a number of hurdles to overcome to make it a
practical reality.

\subsection{Multipath Effect}

As wireless signals can be reflected, refracted and bounced off surrounding objects like walls and furniture in an indoor environment,
multiple copies of the same signal may reach the receiver with different phase delays and power attenuations. This phenomenon is known as
the multipath effect. Multipath makes it difficult to obtain clear signal information (\RSSI, phase, \AoA, \ToF, etc.) which is required
for achieving accurate sensing. While recently literature has proposed to exploit multiple links for cancel the static noises, or select
optimal sub-carriers that are resistant to multipath effect for sensing task, how to make sure the scheme can be portable to different
environments remains an open problem. There are other schemes for dealing with the multipath effect, but they often require a large
bandwidth or antenna array. This assumption reduce the practicability of the approach.


Many wireless sensing methods work by observing, from the receiver side, how the signals sent by the transmitter are affected by the target
object. Often, the reflected signals are weaker compared to those bounced from non-target objects (e.g., walls or furniture), making it
difficult to track the desired signals. Finding a way to cancel the impact of other surrounding objects would be helpful, but existing
solutions generally assume the environment is static (i.e., no other people movement around or no door opening and closing). Therefore,
there is a need for effective multipath and noise suppressing algorithms.


\subsection{Clock Synchronization} Wireless sensing systems often use multiple devices, such as transmitters and receivers, to track
objects and activities. These devices often need to exchange information at a high frequency. Precise clock synchronization is essential
for good performance because the time is used to calculate wireless channel information such as the \ToF, and an error of nanoseconds can
translate to a localization error of a few meters. However, obtaining synchronized time at a fine-grained level among computing multiple
devices is challenging due to the differences in the hardware clocks. One solution is to using a wire to synchronize the clocks between two
devices, but the effectiveness of the solution decreases as the distance between the devices increases. This limits the applicability of
the approach.

\subsection{Deployment Cost} The performance of a wireless sensing system highly depends on how the transmitters and receivers are placed
and the environment the sensing model is tuned for. Determining the optimal device deployment is challenging due to the subtle interactions
of the signal coverage and the multipath effects~\cite{wang2016human}. The optimal deployment also changes if the environment has changed,
e.g., adding or removing some furniture. Such a change often requires recollecting the training samples to update the sensing model, which
is often associated with intensive human involvement at a high cost. A recent work addresses the issue by using machine learning to
generate synthetic training examples~\cite{CrossSense}. While this work was able to reduce the cost for collecting training data, it does
not eliminate it. Therefore, the issue of deployment cost remains.



%\paragraph*{Generalization problem} Many learning based sensing methods can work effectively with even a single transceiver pair.
%This reduces the hardware cost. However, one major unsolved issue is that the learned model can hardly adapt to the changes of environment
%and time. For long-term or varying scenarios, the sensing accuracy would decline seriously if we still utilize the original patterns, and
%constant updating patterns comes with a huge training cost. Recent works attempt to solve this problem by deep learning technology that can
%extract hidden useful features of higher dimensions, while these methods lack theoretical basis. To make sensing more convincing, it is
%better to combine learning with theoretical models by excavating and analyzing the physical nature of signal changes.

%\subsection{Tiny signal extraction} Some interfaces realize their sensing objective by leveraging the reflected signal caused by the
%target. Nevertheless, the wireless signal arriving at receiver is always a superposition of  direct path and various reflected paths. In
%many cases, the target reflected signal is quite subtle compared to the signal from direct path and strong reflectors, so it is very
%challenging to separate the weak component from the mixed signal. Traditional solutions generally suppress undesired paths by prior
%measurement, while they are still unable to thoroughly attain the desired reflected component. The latest work introduce a new idea that
%conducts nonlinear frequency shifting to the signal reflected off target, this makes it different from the other interfering signals, thus
%it can be separated. However, this approach still limited to contact sensing. To achieve tiny signal extraction in more complex non-contact
%scenarios, we believe that the multiple paths approximate estimation and blind-source separation algorithms can serve as the promising
%solutions.

\subsection{Evaluation Standard}
Currently, a proposed sensing technique is typically evaluated on bespoke datasets and the experiment is conducted in specific environments
and settings. Like any engineering subject, the lack of evaluation standards and common datasets make it difficult to evaluate the
effectiveness and generalization ability of different approaches. Many existing approaches in the field are built upon machine learning
techniques. Peer review of a machine learning approach is difficult. The black-boxing mechanism prevents the quality of the model from
being questioned unlike analytical methods. As a result, reviewers now have to scrutinize that the experiments were fairly done. This means
all the implementation and test data must be publicly available for scrutiny.

We see the research community is pushing reproducibility by e.g., encouraging authors to opensource their code and datasets and to
participate in artefact evaluation. We hope such movements can result in some widely accepted evaluation standards and public datasets.


 %Currently, there is no standard set of specifications that can serve as a guide for designing
%sensing techniques. There is no single wireless technology that is widely accepted as the main technology for future sensing systems. As
%evident from our discussion on the proposed systems in the previous sections, a number of different technologies and techniques have been
%used for the purpose. However, most of the systems are disjoint and there is no ubiquitous system that currently exists. This poses
%significant challenges. Therefore, we believe that there is a need for proper standardization of sensing. Through standardization, we can
%set the specifications and also narrow down the technologies and techniques that can satisfy the regular evaluation metrics. Furthermore,
%there is a need for creating a universal benchmarking mechanism for evaluating an sensing system.
%
%In order to facilitate communities to restore the published system and ensure the feasibility of the system, opening data set is becoming a
%trend in the academic community. However, from the authors' point of view, it may be inconvenient to publish all data sets in a time,due to
%the consideration of later research extensions. We suggest that a compromise way is to expose source code and some data sets.

\section{Future Directions}

Wireless sensing constructed around consumer-grade devices have demonstrated its utility as a means of information sensing. It will
continue to be used for new and more complex application contexts.

The open research question goes beyond simple settings such as single-link, single-user scenarios. It would be interesting to see whether
the technique can be applied to more complex scenarios including in-body localization, through-wall imaging, sensing of high-speed mobile
object (high-speed rail, driverless vehicles, UAV), etc. Can we integrate sensing with the communication protocol? By doing so, we might be
able to unlock some advanced application scenarios such as cross-medium, cross-protocol, and cross-frequency sensing. There is a wide range
of interesting research questions related to combining multiple means (e.g., multiple signals, multiple devices, active and passive, model
and deep learning) and multidimensional parameter estimation.

\section{CONCLUSION}\label{sec:8conc}
This article has introduced wireless sensing methods built upon consumer-grade solutions. After much progress has been made in the last
decade or so, wireless sensing is now a mainstream research area and has a large amount of academic interest and papers. While it is
impossible to provide a definitive cataloguer of all relevant research, we have tried to give an accessible tutorial to the main research
topics, challenges and future trends. Wireless sensing is still at its early stage.  There are many challenges ahead, but there are
much greater opportunities for creative and high-impact research to be conducted in this exciting direction.



\footnotesize
\bibliographystyle{IEEEtran}
\bibliography{reference}

\end{document}
