\section{Future Directions}

Wireless sensing constructed around consumer-grade devices have demonstrated its utility as a means of information sensing. It will
continue to be used for new and more complex application contexts.

The open research question goes beyond simple settings such as single-link, single-user scenarios. It would be interesting to see whether
the technique can be applied to more complex scenarios including in-body localization, through-wall imaging, sensing of high-speed mobile
object (high-speed rail, driverless vehicles, UAV), etc. Can we integrate sensing with the communication protocol? By doing so, we might be
able to unlock some advanced application scenarios such as cross-medium, cross-protocol, and cross-frequency sensing. Currently, wireless
sensing is done in an ad hoc manner and requires revising the device driver to obtain channel metrics like \CSI. Future wireless standards
should perhaps make the support of channel metric measurements part of the standard to lower the implementation barrier of the technique.
There is also a wide range of interesting research questions related to combining multiple means (e.g., multiple signals, multiple devices,
active and passive, model and deep learning) and multi-dimensional parameter estimation.
