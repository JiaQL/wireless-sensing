\section{Future Directions}

Wireless sensing constructed around consumer-grade devices have demonstrated its utility as a means of information sensing. It will
continue to be used for new and more complex application contexts.

The open research question goes beyond simple settings such as single-link, single-user scenarios. It would be interesting to see whether
the technique can be applied to more complex scenarios including in-body localization, through-wall imaging, sensing of high-speed mobile
object (high-speed rail, driverless vehicles, UAV), etc. Can we integrate sensing with the communication protocol? By doing so, we might be
able to unlock some advanced application scenarios such as cross-medium, cross-protocol, and cross-frequency sensing. There is a wide range
of interesting research questions related on combining multiple means (e.g., multiple signals, multiple devices, active and passive, model
and deep learning) and multidimensional parameter estimation.
