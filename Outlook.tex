\section{Future Directions}

Wireless sensing constructed around consumer-grade devices have demonstrated its utility as a means of information sensing. It will
continue to be used for new and more complex application contexts.

In view of the state-of-arts and practical applications, we expect the future research in wireless sensing to follow two trends.
\textbf{(i) Towards complex sensing scenarios.} Typical applications include in-body localization, through-wall imaging, sensing of
high-speed mobile object (high-speed rail, driverless vehicle, UAV), etc. \textbf{(ii) Integrating sensing with communication.} By
combining the advantages of sensing and communication, we can achieve more amazing sensing and communication technologies such as
cross-medium, cross-protocol, cross-frequency. We believe the potential solutions would focus on combining multiple means (multiple
signals, multiple devices, active and passive, model and deep learning) and multidimensional parameter estimation.
