\section{Working Mechanism of Wireless Sensing\label{sec:mechanism}}
\begin{figure} [t!]
       \centering
        \subfigure[]{
                \includegraphics[width=0.22\textwidth]{figures/setup_example.pdf}
                \label{fig:scenario_1}
        }
        \subfigure[]{
                \includegraphics[width=0.22\textwidth]{figures//diff.pdf}
                \label{fig:csi_diff}
        }%
     \caption{The working mechanism of wirless sensing. (a) shows a typical set up where two wireless devices are deployed
     to collect the \CSI values when a user walks pass the devices. (b)  shows the measured \CSI  for two
     individuals, where the resulted \CSI amplitudes are relatively consistent for the same person across multiple walks, but are sufficiently different for different people.}
     \label{fig:csi_demo}
\end{figure}


To illustrate how wireless sensing works, consider the gait identification scenario depicted in Figure~\ref{fig:scenario_1}. The sensing
task in this example is to identify which of a set of known users has walked pass the scene. Knowing this information allows one to -- for
example -- personalize the light setting of a smart home. Here, two wireless routers (a sender and a receiver) are used to measure how the
user's movement affect the wireless channel metric, such as the channel state information (\CSI) or received signal strength indicator
(\RSSI). The idea is that the wireless signal will bounce off the wall, furniture, and human body, and the unique pattern of a person’s
activity (walking in this example) would lead to a unique pattern in the wireless signal; by measuring how the wireless signal is affected
by the human activity and comparing the measurement against some pre-collected fingerprints or training data, one can infer what activity
has been performed and by whom.

Figure~\ref{fig:csi_diff} shows the measured \CSI amplitudes of two users in our scenario. In this case, each user walked through our scene
five times and the \CSI amplitude per walk is shown. This figure suggests that WiFi channel metrics can be a useful means for user
identification, because the channel metric measurements for the same person is relatively consistent but is sufficiently different for the
two  people.
