\section {Wireless Sensing Signals}
A wide range of wireless signals have been used in the past for various sensing tasks. Table~\ref{Tab1} summarizes what wireless signals
have been employed in previous consumer-grade solutions.


\renewcommand\arraystretch{2}
\begin{table*}\scriptsize
%\centering
\caption{Consumer-grade wireless signals used in prior sensing tasks.}
\label{Tab1}
\setlength{\tabcolsep}{7mm}
\begin{tabular}{p{0.5cm}p{0.6cm}<{\raggedright}p{0.9cm}<{\raggedright}p{1.0cm}<{\raggedright}p{0.6cm}<{\raggedright}p{1.2cm}<{\raggedright}p{0.9cm}<{\raggedright}p{1.2cm}<{\raggedright}}
\toprule
%\diagbox[width=5em,trim=l]{Properties}{Signal} & Wi-Fi & RFID & UWB & Asoustic & LoRa & FMCW radar & Visible Light \\
\textbf{Properties} & \textbf{WiFi} & \textbf{RFID} & \textbf{UWB} & \textbf{Asoustic} & \textbf{LoRa} & \textbf{FMCW radar} & \textbf{Visible Light} \\
\midrule
\rowcolor{Gray} \textbf{Range} & 35m & 12m & 10-20m & 2-3m & 15Km & 9m-120km & 1.4Km\\
\textbf{Frequency} & 2.4GHz/5GHz & 902.75-927.25MHz & 3.1-10.6GHz & 17-24KHz & 868MHz/903-927.5MHz & 24-24.25GHz & 380-790THz\\
\rowcolor{Gray} \textbf{Bandwidth} & 20/40MHz & 24.5MHz & 1GHZ & -- & 125/250/500KHz & 250MHz & --\\
\textbf{Metrics} & CSI, \RSSI & Phase, Doppler, \RSSI & Phase, \RSSI& Phase, \RSSI & Frequency, Phase, \RSSI & Frequency, Phase, \RSSI & \RSSI\\
\rowcolor{Gray} \textbf{Cost} & Low & Low & High & Low & Low & High & High\\
\textbf{Modulation} & OFDM & OOK & PPM & AM/FM/PM & CSS& FMCW & OOK/CSK/VPPM\\
\bottomrule
\end{tabular}
%\caption{Different wireless signals with different properties.}%\lable{Tab1}
\end{table*}

%From the Table~\ref{Tab1}, we can see that different wireless signals not only have their own unique properties, but also share the common
%information including phase, \RSSI and frequency. These common information allow them to be applied for sensing, and their own
%characteristics determine their own befitting sensing scenarios.

\FIXME{Here we need to give an extensive discussion on the prosperities of different wireless signals.}

Which signal is best for a sensing task, is the \$64,000 question. The answer is, it depends. Some wireless signals like LoRa  may provide
a longer sensing range, but they require specialize hardware. \FIXME{Here we contrast and discuss the strengths and weakness of different
wireless signals. }
