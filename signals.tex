\subsection {Choices of Wireless Sensing Signals}
As summarized in Table~\ref{Tab1}, a wide range of wireless signals have been employed in prior work for information sensing. These include
WiFi, \RF, acoustic, LoRa and visible lights, frequency-modulated continuous-wave (\FMCW) radar and ultra-wideband (\UWB). Signals like
WiFi, acoustic, and visible lights are readily available on commercial off-the-self devices (such as smartphones and smart light bulbs),
while others like \UWB, LoRa and \FMCW often require specialize hardware to operate.



\renewcommand\arraystretch{2}
\begin{table*}\scriptsize
%\centering
\caption{Consumer-grade wireless signals used in prior sensing tasks.}
\label{Tab1}
\setlength{\tabcolsep}{7mm}
\begin{tabular}{p{1cm}p{0.6cm}<{\raggedright}p{0.9cm}<{\raggedright}p{1.0cm}<{\raggedright}p{0.6cm}<{\raggedright}p{1.2cm}<{\raggedright}p{0.9cm}<{\raggedright}p{1.2cm}<{\raggedright}}
\toprule
%\diagbox[width=5em,trim=l]{Properties}{Signal} & Wi-Fi & RFID & UWB & Asoustic & LoRa & FMCW radar & Visible Light \\
\textbf{Properties} & \textbf{WiFi} & \textbf{RFID} & \textbf{UWB} & \textbf{Asoustic} & \textbf{LoRa} & \textbf{FMCW radar} & \textbf{Visible Light} \\
\midrule
\rowcolor{Gray} \textbf{Communication Range} & 35m & 12m & 10-20m & 2-3m & 15Km & 9m-120km & 1.4Km\\
\textbf{Frequency} & 2.4GHz/5GHz & 902.75-927.25MHz & 3.1-10.6GHz & 17-24KHz & 868MHz/903-927.5MHz & 24-24.25GHz & 380-790THz\\
\rowcolor{Gray} \textbf{Bandwidth} & 20/40MHz & 24.5MHz & 1GHZ & -- & 125/250/500KHz & 250MHz & --\\
\textbf{Metrics} & CSI, \RSSI & Phase, Doppler, \RSSI & Phase, \RSSI& Phase, \RSSI & Frequency, Phase, \RSSI & Frequency, Phase, \RSSI & \RSSI\\
\rowcolor{Gray} \textbf{Cost} & Low & Low & High & Low & Low & High & High\\
\textbf{Modulation} & OFDM & OOK & PPM & AM/FM/PM & CSS& FMCW & OOK/CSK/VPPM\\
\bottomrule
\end{tabular}
%\caption{Different wireless signals with different properties.}%\lable{Tab1}
\end{table*}


Which signal is best for a sensing task, is the \$64,000 question. The answer is, it depends. For examples, while WiFi is almost
everywhere, it is limited in the communication range (i.e., 5 to 10 meters) and consumes a significant amount of power; by contrast, LoRa
supports a longer range (i.e., 15 Kilometers) and needs significantly less power to operate, but it has a much lower bandwidth and is more
sensitive to the environmental multipath. Different signals also operate at different frequencies and offer a variety of sensing metrics.
Some of the signals like \FMCW radar can provide an accurate tracking capability but is more expensive to deploy. It is important to note
that there is no ``one-size-fits-all" sensing medium, and the choice depends on the trade-off between the precision requirements, the
deployment environment, and the budget.
