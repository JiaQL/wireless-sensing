\subsection {Choices of Wireless Sensing Signals}


As summarized in Table~\ref{Tab1}, a large variety of wireless signals have been employed in prior work for information sensing. These include
\WiFi, \RFID, acoustic, \LoRa, visible lights, 60GHz and ultra-wideband (\UWB). Signals like
WiFi, acoustic, and visible lights are readily available at COTS devices such as smartphones and light bulbs,
while others like \UWB and 60GHz often require dedicated hardware to operate.



Which signal is best for a sensing task, is the \$64,000 question. The answer is, it depends. For example, while \WiFi is ubiquitous indoors, it is limited in the communication range~(15-50m) and consumes a significant amount of power. In contrast, \LoRa
supports a much longer range up to 15km and needs much less power to operate, but it has a much smaller bandwidth and is more
sensitive to multipath and interference. %Different technologies also operate at different carrier frequencies and offer a variety of sensing metrics.
On the other hand, acoustic signals can achieve a surprisingly high sensing accuracy at millimetre level~\cite{LLAP} due to the much smaller propagation speed in the air. However, the sensing range of acoustic signals is usually below 1m.
%Some of the signals like \FMCW radar can provide an accurate tracking capability but is more expensive to deploy.
It is thus important to note
that there is no ``one-size-fits-all" sensing signal, and the choice depends on the trade-off between the precision requirements, the
deployment environment, and the budget.
