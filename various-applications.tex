\section{Wireless Sensing Applications}
Wireless sensing has been demonstrated to be effective on a wide range of applications from localization, activity recognition to material
identification.


\subsection{Localization} Target localization and tracking are two important sensing applications. \WiFi and \RFID are the popular technologies used for this task, due to the low-cost and wide deployment of the hardware.
Acoustic signals are also employed for high-accuracy tracking in a small range. % thanks to its slow propagation velocity and robustness to environment changes.

Localization can be achieved by using wireless fingerprints or predictive models. A fingerprint-based approach works
by first collecting the wireless fingerprints~(e.g., \RSSI or \CSI) at chosen locations. During deployment, it compares the measured CSI/RSSI readings against the pre-collected fingerprints to infer the
location of the target object. Such an approach is easy to be deployed but requires high human-labor cost and its accuracy is susceptible to the environmental changes such as furniture movement.

As an alternative, predictive models learn the correlation between the wireless signal characteristics and the  locations. These
approaches can be categorized into two groups. The first is to use the arrival-of-angle (\AoA) or time-of-flight (\ToF) of the wireless
signals for location estimation. The second is to exploit the signal amplitude for propagation distance estimate. Predictive models avoid the fingerprint collection phase but still requires hardware calibration such as phase calibration among RF front ends~(for \AoA) and time synchronization between the transceiver pair~(for \ToF).  Besides, the accuracy of the \AoA based methods depends on the size of the antenna array at the receiver while the frequency bandwidth is the key determining factor for \ToF based methods. UWB with a wide frequency band is thus an ideal candidate for \ToF-based localization and tracking.     %have the advantage of  %a better generalization ability over the fingerprint counterparts, i.e., it can estimate arbitrary locations of the target environment.

%However, predictive models are not a panacea and often suffer from hardware imperfections (e.g., it is difficult to obtain an accurate
%\ToF) and multipath effects (which make it hard for obtaining good signal attenuation). They also have the difficulties for determining the
%target's initial position. Several countermeasures have been proposed to address these issues. These include (i) using pre-collected
%fingerprints~\cite{Wang2016D}, directly modeling the phase differences~\cite{LLAP},  information obtained at multiple signal
%frequencies~\cite{RFind, CAT, Strata}, or multiple tracking devices~\cite{BeepBeep} to improve the information gain; (ii) using multiple
%antennas~\cite{Arraytrack, Spotfi} to cancel the multipath effect or directly modeling the multipath profiles~\cite{PinIt}; and (iii)
%leveraging the virtual antenna array to obtain the initial position~\cite{Tagoram}.





\subsection{Activity and Behavior Recognition}
Human activity and behavior recognition is another area where wireless sensing has demonstrated a great potential.

Some of the early work in the area use multiple transmitters and receivers to construct a 3D lattice of wireless links to identify if there
are human movement. The latest efforts have been made to use a single device equipped with multiple antennas to realize activity recognition. For examples, WiSee~\cite{WiSee} was able to differentiate 9 commonly seen body gestures with the help of machine learning methods. %WiTrack shows that it is possible to recognize hand gestures from signal perturbations.
A more recent work illustrates the possibility of using
wireless signals for measuring human's breathing and heart rate~\cite{Smart-homes}. The idea is that the chest movement caused by
breathing would change the signal reflections, which in turn allows one to capture the breathing event and count the heart rate at high enough
accuracies. By precisely counting the heart beats, one can even predict a user's emotion~\cite{Zhao2017Emotion}.

Machine learning based classifiers have been widely used in activity recognition. The effectiveness of these methods in general depends on
the quality of the training data and features for capturing the characteristics of the activity. As a result, some of the recent research
efforts turn into seeking ways to automatically identify useful features~\cite{CrossSense} or reduce the human involvements (and thus the
 labor cost) for generating high-quality training data~\cite{zhao2018rf}.





\subsection{Target Imaging and 3D Reconstruction}

Target imaging is another area that has attracted extensive attentions. Traditional imaging systems like Radar, X-rays and CT employ dedicated
hardware with high frequency, large bandwidth and antenna arrays, which are extremely expensive and usually large in size. In recent years, researchers attempt to employ \RF signals generated from cheap commodity device to image a target's shape. For instances, WiFi based system Wision~\cite{Huang2014Feasibility} utilizes the re?ected signals from the target for imaging. Its imaging performance is quite coarse due to the small bandwidth of commodity WiFi. Tagscan~\cite{Tagscan} leverages the RFID signal change at each angle to obtain the target width information and stitches together the width information at many angles to obtain the horizontal cut image of the target. However, this approach needs a robot with antenna moves around the target, which extremely limits the system's practicability. 

To make compromise between hardware cost and accuracy. New methods by using \FMCW technology for imaging have been proposed. RF-Capture~\cite{Adib2017Capturing} can reconstruct the coarse skeleton of a human target behind a wall by extracting the signal reflections from body parts of a moving target. The latest work RF-Pose3D~\cite{zhao2018rf} infers 3D human skeleton by incorporating \RF signal~(\FMCW) and visual stream. It first leverages deep CNN to infer
the person's 3D skeleton, and then uses the visual stream to label different target skeletons. RF-Pose3D can track each key point
on the human body at a centimeter-level accuracy.


\subsection{Object Material Identification}
Material identification is another exciting direction for wireless sensing. It can be adopted in
applications such as detecting concealed weapons and expired milk. % deterioration.

Existing material identification systems such as Radar, X-rays, CT, MRI and B-scan ultrasonography employ extreme high
frequency signals for sensing. They are expensive and need professionals to operate. Some of the recent work utilize the low-cost
wireless signals like \RFID to sense the target material. TagScan~\cite{Tagscan} detects the changes of the
\RSSI and phase when the \RF signal penetrates through the target. When evaluating on 10 liquids, TagScan is able to successfully identify
the test objects at 94\% accuracy and differentiate between Coke and Pepsi.% Such an approach offers a low-cost alternative, but due to the limitation of the technology, it is not expected to be used at security-critical scenarios like airport security check.


\subsection{Vibration Detection}   Vibration detection plays a vital role in many applications, such as malfunction detection in electronic instrument and
monitoring the displacement in vehicle, etc. Traditional vibration sensing schemes~\cite{Lion} require dedicated sensors, which are very expensive. And
most of them have a limit monitoring area.

To avoid these drawbacks, TagBeat~\cite{Tagbeat} proposes to utilize commodity \RFID to achieve sub-millisecond vibration sensing. In order
to magnify tiny vibration signal induced by micro-vibration, TagBeat exploits a special signal processing technique to virtually
extend the vibration radius, which is equivalent to multiplying vibration amplitude by a constant. Note that the side-effect of magnification is that noise is amplified
too. TagBeat also proposes a compressive sensing model to reconstruct the vibration signal to overcome the challenge of detecting high-frequency vibration with a limited sampling rate.

