\section {Various sensing applications}

In recent years, with the emerging of the Internet of Thing (IoT), many wireless sensing applications have been inspired, especially for
localization and tracking, behavior perception, material identification, target imaging and vibration detection. In this section, we will
present these five mainstream applications respectively. We also discuss the challenges coming with them and corresponding efforts have
made.

\subsection{Localization and Tracking} Target localization and tracking are two most universal and significant sensing applications.
Specifically, indoor localization has attracted much interest from both the industry and the research community. Among all the technologies
employed for localization, Wi-Fi and RFID are considered as the most promising schemes due to the cheap price and widespread deployment.
Furthermore, we can easily access the amplitude and phase information from commercial RFID and Wi-Fi devices. In the following, we will
discuss previous indoor localization works, which can be classified into two categories: learning-based and model-based localization
schemes.

\textbf{Learning-based localization:} These systems collect received signal strength~(RSS) and channel state information~(CSI) at each
location to build fingerprints for localization, which can achieve a meter-level accuracy. Although these systems are easy to be deployed,
their localization accuracies are susceptible to environmental change and indoor multi-path. Furthermore, learning-based localization
schemes suffer from labor-intensive offline training.

\textbf{Model-based localization:} For a robust, higher accuracy and low cost system, more and more researches have a trend to explore
model-based localization approaches, which can be divided into two categories. The first category of localization systems are based on
arrival-of-angle~(AoA) or time-of-flight~(TOF) estimation, which achieve localization by intersecting multiple AoA or ToF estimates. The
second category is energy-attenuation-based localization systems, which establish signal energy attenuation model to calculate target
localization. These ideas sound simple, however, there exist some common challenges to make them practical.

\paragraph*{Hardware imperfections} Due to commercial hardware imperfections, we cannot obtain accurate phase information and accurate time of
flight.

\paragraph*{multi-path effect} In our indoor environment, there exits rich  multi-path effects, which result in the inaccuracy of location
parameter estimation and signal attenuation model construction.

\paragraph*{Uncertainty of initial position} For the tracking systems, it is difficult to determine target's initial position.



 To mitigate the phase error caused by commercial hardware imperfections, a general idea is to calibrate phase information by comparing it
 with the real pre sampled values \cite{Wang2016D}. Meanwhile, different frequencies can be utilized to emulate a very large bandwidth for
 achieving accurate time estimation~\cite{RFind}. To eliminate the multi-path effect, on the one hand, we can conversely exploit multi-path
 to capture multi-path profiles for target localization~\cite{PinIt}. On the other hand, by increasing the number of antennas in the array,
 we can improve angle estimation resolution to identify the line-of-sight~(LoS)~\cite{Arraytrack, Spotfi}. Furthermore, one skillful idea
 is to select clear subcarrier information to establish signal attenuation model~\cite{wang2016lifs}. To deal with the uncertainty of
 initial position, we can leverage virtual antenna array and approximate evaluation method to improve tracking accuracy~\cite{Tagoram}.

In addition, acoustic signal has been widely used for localization and tracking, due to its slow propagation velocity and robustness to
environment changes. However, there also exist several challenges that are similar to Wi-Fi and RFID based localization systems.

The first issue need to be tackled is the inaccuracy phase information and unsynchronized time induced by the defects of commercial
acoustic devices. A standard solution is to use phase difference for tracking target location~\cite{LLAP}, another approach is employing
multiple device to avoid the time asynchrony~\cite{BeepBeep}.  The second issue is the indoor multi-path, which results in inaccuracy of
model parameters estimation. To deal with this problem, a typical intuition is to design different modulation for acoustic signal, such as
frequency modulated carrier wave~(FMCW) and orthogonal frequency division multiplexing~(OFDM) \cite{CAT,STRATA}.

\subsection{Behavior sensing} Human behavior recognition can enable a wide variety of applications, such as heath care, fitness tracking and sleep monitoring. Similarly to localization and tracking systems, human behavior recognition interfaces can also be classified as learning-based and model-based behavior recognition two categories.

\textbf{Learning-based behavior recognition:} For most irregular behaviors recognition, researches depend on pattern learning (DTW~\cite{Wang2014We}, KNN~\cite{Forster2011Incremental}, etc) to deal with it. To ensure the recognition accuracy of systems, we need to answer following questions.

The first question is how to design a high-quality feature to achieve cross-site and large-scale sensing? To answer this, a intuitive idea is performing ensemble learning to extract the unique feature of each target~\cite{CrossSense}. The other question is how to effectively label substantial data? To address this, recent work proposes to borrow vision signal to help achieve labeling task in wireless sensing ~\cite{zhao2018rf}.

\textbf{Model-based behavior recognition:} These model-based behavior recognition schemes can only be applied to analyze few human behaviors with a specific frequency range, such as coarse movements--walking, jogging, running~\cite{Wang2015Understanding} and fine-grained breathing and heart beat~\cite{Smart-homes}. These activities can be sensed according to that they can cause signal fluctuates with specific frequencies. However, the common challenge is how to eliminate the interference from the other body parts' activities? To address this, a general idea is to filter out the undesired activities beyond the target signal frequency range in frequency domain.

\subsection{Material identification} Target material identification is a key role in our life, many applications would benefit from it, such as detecting concealed weapons and determining food deterioration. Existing material identification systems such as Radar, X-Ray, CT, MRI and B-scan ultrasonography use special hardware with high frequency, large bandwidth and antenna arrays, which are extremely expensive and usually large in size. Hence, there exits some challenges to identify target material with commodity device.

How to eliminate the impact of target's size? How to remove the impact of containers and liquid reflections?

To solve the first challenge, we can design a feature, which only depends on the material type and independent of target size, such as the ratio of RSS change and phase change \cite{Tagscan}. Consider that the liquid containers and liquid reflections have some impacts on the above feature, \cite{LiquID} employs UWB devices with high bandwidth to accurate estimate the time of signal propagation, and use it to remove the impact of containers and liquid reflections.


\subsection{Target imaging and 3D reconstruction} Wireless signal can reflect off or penetrate through various objects in the environment. Thus researches attempt to utilize this to achieve object imaging.

\textbf{Reflection-based imaging:} Due to the physics of radio reflections, at any point in time, the sensor captures signal reflection from only a subset of the human body parts. \cite{Adib2017Capturing} identifies human body parts from RF snapshots across time to capture the human figure, as the consecutive RF snapshots can expose different body parts and diverse perspectives of the same body part; However, imaging results are highly sensitive to phase accuracy. Thus~\cite{Zhu2015Reusing}  proposes a new 60Ghz imaging algorithm, which images an object using only RSS recorded along the device��s trajectory, as the RSS measurements of 60Ghz are highly robust against noises in device position and trajectory tracking.~\cite{mao2018aim} proposes a new Phase Gradient Algorithm called MPGA to remove the impact of hand jitters, compensate phase errors, MPGA can effectively support acoustic imaging in a mobile context.

Occlusion is a fundamental problem in human pose estimation. Thus, \cite{Zhao_2018_CVPR}  presents a different approach that extracts pose information from the visual stream, and uses it to label the RF signals for dealing with Occlusion. To estimate the 3D skeletons, we need to use RF signals to extract full 3D skeletons of people including the head, arms, shoulders, hip, legs, etc., and apply the convolutional neural network (CNN) to 4D convolution, but common deep learning platforms (e.g., Pytorch, TensorFlow) do not support 4D CNNs. Therefore,  \cite{zhao2018rf}leverages the properties of RF signals to performs 4D convolutions by decomposing them into a combination of 3D convolutions performed on two planes and the time axis.

\textbf{Transmission-based imaging:} Each propagation distance inside a target represents one piece of target width information at one angle, by stitching together the width information at many angles can obtain the horizontal cut image of the target, but stitching them together to create the final image is still challenging, as the starting points of the propagation distances are unknown. \cite{Tagscan}discovers two target images estimated by two arrays will align well when the starting points of propagation distances are correctly selected, and thus model the imaging problem as an optimization problem by minimizing the difference of two images estimated from the two arrays.


\subsection{Vibration detection} Vibration detection plays a vital role in many applications, such as malfunction detection in electronic instrument, monitoring the displacement in vehicle, etc. Traditional vibration sensing schemes require dedicated sensors, which are very expensive. And most of them have a limit monitoring area. To avoid these drawbacks, \cite{Tagbeat} proposes to utilize commodity RFID to achieve
vibration sensing. However, the challenges are: how to magnify tiny vibration signal induced by micro-vibration for easier detection? how to detect high-frequency vibration with a limited sampling rate of RFID?

For the first challenge, a simple mathematical methodology is introduced, which multiplies the vibration amplitude by a constant. To deal with the second problem, a compressive sensing model is used to reconstruct the vibration signal so that can achieve sub-millisecond accuracy in vibration sensing.
